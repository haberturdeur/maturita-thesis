\chapter{Instalace a nastavení SolidWorks}
\section{Instalace SolidWorks SDK}

\subsection{Stažení instalátoru a získání licenčních klíčů}
Začneme otevřením webové stránky \href{http://www.solidworks.com/sdk}{www.solidworks.com/sdk}.
Zobrazí se nám formulář, do kterého vyplníme údaje o sobě (jméno, příjmení, e-mail a status - student).
Je nutné psát \B{bez diakritiky}!

\fxnote[author=PŠ]{\textcolor{mygreen}{Sem přijde screenshot formulářes}}

V sekci \B{Product information} pod textem \B{„I already have a Serial Number that starts with 9020“} zaškrtneme možnost \B{No} a do kolonky níže napíšeme \B{9SDK2019}.
Na pravé straně poté zaškrtneme nejnovější verzi, tedy \B{2020-2021}.
Vyplněný formulář odešleme kliknutím na tlačítko \B{Request download}.
Na další stránce potvrdíme licenční podmínky tlačítkem \B{Accept and Continue}.

Nyní jsme se již dostaly na stránku, odkud můžeme SDK stáhnout.
Klikneme tedy na tlačítko \B{Download}, čímž si stáhneme instalátor.
Okno ještě \B{nezavíráme} - budeme z něj potřebovat zkopírovat licenční čísla. 

\subsection{Instalace}
Stažený instalátor otevřeme. 
Objeví se nám okno, ve kterém můžeme nastavit, kam chceme vyextrahovat soubory instalace.
Jakmile máme umístění zvolené, klikneme na tlačítko \B{Unzip}.
Chvíli počkáme a otevře se nám \It{Manažer instalací SOLIDWORKS 2020}.
Pokud se nám objeví okno informující, že po předchozí instalaci nebyl dokončen restart systému, stačí jej odklepnout tlačítkem \B{OK}.
Na obrazovce, kde můžeme zvolit typ instalace ponecháme zaškrtnuté \It{Instalovat na tento počítač} a klikneme na \B{Další}.

Nyní po nás bude instalátor chtít zadat sériová čísla. 
Otevřeme si tedy webový prohlížeč se stránkou, kde byla tato čísla napsaná.

\section{Instalace školních šablon a norm. dílů}

\subsection*{Stažení .ZIP archivu}

\subsection*{Instalace šablon a knihoven materiálů}

\subsection*{Instalace normalizovaných dílů}


\section{Zprovoznění RealView na necertifikovaném počítači}
\subsection*{Co je to režim RealView?}
Režim zobrazení RealView umožňuje věrnější zobrazení modelů díky vylepšenému stínování a odleskům.
Tento režim je ale podporován jen relativně malým počtem certifikovaných grafických karet NVIDIA Quadro a Radeon Pro.
Aktivace na ostatních grafických kartách je možná s malým zásahem do registru.\newline

\noindent\textcolor{red}{VAROVÁNÍ: Při aktivaci budeme zasahovat do registru systému, je tedy nutné se přesně řídit návodem. Zásah v registru na špatném místě může způsobit nestabilitu operačního systému, nebo aplikací.}

\subsection*{Zjištění označení aktuální grafické karty}
Než začneme cokoliv dělat, musíme zkontrolovat, že je SolidWorks vypnutý. 
Pokud ne, hned tak učiníme.
Na klávesnici zmáčkneme klávesovou zkratku \B{Win + R}, otevře se nám dialog \It{Spustit}.
Do políčka napíšeme \It{regedit} a potvrdíme Enterem.
Kliknutím na tlačítko \It{Ano} potvrdím udělení administrátorských oprávnění v okně \It{UAC}.

V levé části editoru registru postupně proklikáváme složky \newline HKEY\_CURRENT\_USER $>$ SOFTWARE $>$ SolidWorks $>$ SOLIDWORKS 2020 $>$ Performance $>$ Graphics $>$ Hardware $>$ Current.
Při kliknutí na poslední složku se nám vpravo objeví několik hodnot, klikneme dvakrát na \It{Renderer}.
Otevře se nám tabulka nastavení hodnoty, za pomoci \B{Ctrl + C} si její údaj celý zkopíruji (např. \It{GeForce GTX 1050/PCIe/SSE2}).

\subsection*{Přidání vlastního klíče do registru}
V levé straně editoru registru nyní otevřu složku \It{GI2Shaders}.
Následně si podle toho, jakou mám grafickou kartu vyberu složku \It{Other} (pokud mám graf. procesor Intel HD Graphics), nebo \It{NV40} (cokoliv ostatního) -- obě jsou obsaženy ve složce \It{GI2Shaders}.
Na zvolenou složku (Other, nebo NV40) kliknu pravým tlačítkem a vytvořím \It{nový klíč}, do jehož názvu vložím hodnotu, kterou jsem si před chvílí zkopíroval za pomoci \B{Ctrl + V}.
Zkontroluji, že je nový klíč vybraný a na pravé straně editoru registru kliknu opět pravým tl. myši.
Tentokrát vytvořím novou \It{Hodnotu DWORD (32 bitová)}, kterou nazvu \It{Workarounds}.
Na novou hodnotu dvakrát poklepu myší a do políčka \enquote{\It{Údaj hodnoty}} napíšu \B{4000080} pro verzi SolidWorks 2020.
Verze 2019 má tento kód lehce odlišný -- \B{30408}.

\subsection*{Vyzkoušení, zda nám RealView funguje}
Teď již jen musíme vyzkoušet, zda nám RealView funguje jak má.
Otevřeme SolidWorks a v něm nějaký díl, nebo sestavu.
Nahoře klikneme na tlačítko se symbolem oka a pokud se mezi možnostmi objeví i RealView, vše je v pořádku. 

%%% Výkresovka - textové přepisy návodů
\chapter{Výkresová dokumentace - vybrané návody}


%%% Vybrané text. návody z modelování
\chapter{Modelování - vybrané návody}

%%% Drážka pro pero v náboji
\section{Drážka pro pero v náboji}

\subsection*{Skica}

\subsection*{Odebrání vysunutím}

%%% Drážka pro pero - hřídel
\section{Drážka pro pero na hřídeli}

\subsection*{Vytvoření roviny}

\subsection*{Skica}

\subsection*{Odebrání vysunutím}

%%% Čelní ozubené kolo s přímým ozubením - modelované splajnem
\section{Čelní ozubené kolo s přímým ozubením}

\subsection*{Vytvoření základního válce}

\subsection*{Profilová skica zubu}

\subsection*{Přidání vysunutím}

\subsection*{Zkosení a zaoblení}

%%% Řetězové kolo
\section{Řetězové kolo}

\subsection*{Vytvovření základního \enquote{talíře} - přidání rotací}

\subsection*{Skica}

\subsection*{Odebrání vysunutím}

\newpage