\chapter{Textové verze vybraných návodů}
\section{Instalace SolidWorks SDK}

\subsection{Stažení instalátoru a získání licenčních klíčů}
Začneme otevřením webové stránky \href{http://www.solidworks.com/sdk}{www.solidworks.com/sdk}.
Zobrazí se nám formulář, do kterého vyplníme údaje o sobě (jméno, příjmení, e-mail a status - student).
Je nutné psát \B{bez diakritiky}!

\fxnote[author=PŠ]{\textcolor{mygreen}{Sem přijde screenshot formulářes}}

V sekci \B{Product information} pod textem \B{„I already have a Serial Number that starts with 9020“} zaškrtneme možnost \B{No} a do kolonky níže napíšeme \B{9SDK2019}.
Na pravé straně poté zaškrtneme nejnovější verzi, tedy \B{2020-2021}.
Vyplněný formulář odešleme kliknutím na tlačítko \B{Request download}.
Na další stránce potvrdíme licenční podmínky tlačítkem \B{Accept and Continue}.

Nyní jsme se již dostaly na stránku, odkud můžeme SDK stáhnout.
Klikneme tedy na tlačítko \B{Download}, čímž si stáhneme instalátor.
Okno ještě \B{nezavíráme} - budeme z něj potřebovat zkopírovat licenční čísla. 

\subsection{Instalace}
Stažený instalátor otevřeme. 
Objeví se nám okno, ve kterém můžeme nastavit, kam chceme vyextrahovat soubory instalace.
Jakmile máme umístění zvolené, klikneme na tlačítko \B{Unzip}.
Chvíli počkáme a otevře se nám \It{Manažer instalací SOLIDWORKS 2020}.
Pokud se nám objeví okno informující, že po předchozí instalaci nebyl dokončen restart systému, stačí jej odklepnout tlačítkem \B{OK}.
Na obrazovce, kde můžeme zvolit typ instalace ponecháme zaškrtnuté \It{Instalovat na tento počítač} a klikneme na \B{Další}.

Nyní po nás bude instalátor chtít zadat sériová čísla. 
Otevřeme si tedy webový prohlížeč se stránkou, kde byla tato čísla napsaná.


\newpage