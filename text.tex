\documentclass{template/template}

\usepackage{subcaption}
\usepackage{amsmath}
\usepackage{enumitem}
\usepackage{hyperref}
\usepackage{gensymb} % balíček symbolů
\usepackage{booktabs}
\usepackage[T1]{fontenc} % evropské uvozovky
\usepackage{csquotes} % text lze uvést do uvozovek pomocí \enquote{text}

\usepackage[toc,page]{appendix}
\usepackage{color} % balíček pro obarvování textů
\usepackage{xcolor}  % zapne možnost používání barev, mj. pro \definecolor
\definecolor{mygreen}{RGB}{0,153,153} % nastavení barev odkazů 
\usepackage{listings} % balíček pro formátování zdrojových kódů 
\usepackage[author=,status=draft]{fixme} % vkládání poznámek  
% dva módy (status): draft (poznámky se zobrazují v PDF) / final (poznámky se nezobrazují v PDF)
\usepackage{multirow}
\usepackage{float}

\lstset { %
    language=C++,
    backgroundcolor=\color{black!5}, % set backgroundcolor
    basicstyle=\footnotesize,% basic font setting
}

\addbibresource{text.bib}
\nocite{*}

\titlecz{Využití videonávodů pro výuku konstrukce v SolidWorks} % Název práce
\titleen{Videoguides usage in SolidWorks construction education} % Anglický název práce
\author{Petr Štourač} % Jméno autora
\institution{STŘEDNÍ PRŮMYSLOVÁ A VYŠŠÍ ODBORNÁ ŠKOLA BRNO, Sokolská} % Celý název instituce
\institutiontype{příspěvková organizace} % Typ instituce
\thesistype{Maturitní práce}  % Typ práce/dokumentu
\mentor{Ing. Václav Zavadil} % Jméno vedoucího práce
\mentorstatement{Ing. Václava Zavadila} % Jméno vedoucího práce ve čtvrtém pádě
\field{Strojírenská konstrukce} % Okruh, nebo téma

\placefooter{Brno 2021}

%\usepackage{hyperref} % balíček pro hypertextové odkazy
% \url{www.odkaz.cz}
% \href{http://www.odkaz.cz}{Text který bude jako odkaz}
% \hyperlink{label}{proklikávací_text} - odkaz na text 
% \hypertarget{label}{cíl_odkazu} - cíl odkazu 

\begin{document}
\hyphenation{SOLIDWORKS Solid-Works}

\maketitle

\makecopyrightstatement{V~Brně}

\makethanks{Duis ante orci, molestie vitae vehicula venenatis, tincidunt ac pede. Nemo enim ipsam voluptatem quia voluptas sit aspernatur aut odit aut fugit, sed quia consequuntur magni dolores eos qui ratione voluptatem sequi nesciunt. Proin pede metus, vulputate nec, fermentum fringilla, vehicula vitae, justo. Pellentesque arcu. Cum sociis natoque penatibus et magnis dis parturient montes, nascetur ridiculus mus. Fusce aliquam vestibulum ipsum. Aenean placerat. Excepteur sint occaecat cupidatat non proident, sunt in culpa qui officia deserunt mollit anim id est laborum. Vivamus porttitor turpis ac leo. Nullam eget nisl. Maecenas libero. Nunc auctor. Nullam rhoncus aliquam metus. Integer pellentesque quam vel velit.}

\pagestyle{empty}

\section*{Anotace}
Sem patří anotace v češtině.

\subsection*{Klíčová slova}
5 a příp. více klíčových slov

\vspace{20mm}

\section*{Annotation}
Here goes english version of thesis annotation.
\fxnote[author=PŠ]{Nějak takto vypadá poznámka vytvořená přes fxnote}

\fxnote[author=PŠ]{\textcolor{mygreen}{A dokonce je lze obarvit!}}

\subsection*{Keywords}
Here goes 5 or more keywords

\newpage
\pagestyle{plain}

\tableofcontents % vysází obsah

%%% Začátek práce
\setcounter{figure}{0}
\setcounter{table}{0}
\newpage

% Uvod prace
\chapter*{Úvod}
\addcontentsline{toc}{chapter}{Úvod}
Představte si (alespoň pro mne) klasickou situaci: 
Blíží se termín odevzdání projektu do konstrukčního cvičení.
Jeden ze studentů vyrábí modely v SolidWorks, když v tom najednou se zasekne na nějakém (byť primitivní) prvku, nebo chybě.
Napadne ho, že zná nějakého spolužáka, který nemá s modelováním problém, nebo jej dokonce baví.
Spolužák mu samozřejmě ochotně poradí a student může svůj projekt dokončit.

Nyní si prosím představte situaci, kdy jste ten spolužák.
Ovšem tentokrát s rozdílem, že Vám nepíše jeden student, ale třeba 20 a to za jeden den.
Také z toho již po chvíli začínáte šílet?

\newpage


% Uspory dosazene pouzitim protoplantu
\chapter{Inspirace k tématu}


\newpage

% Uspory dosazene pouzitim protoplantu
\chapter{Textové verze vybraných návodů}
\section{Instalace SolidWorks SDK}

\subsection{Stažení instalátoru a získání licenčních klíčů}
Začneme otevřením webové stránky \href{http://www.solidworks.com/sdk}{www.solidworks.com/sdk}.
Zobrazí se nám formulář, do kterého vyplníme údaje o sobě (jméno, příjmení, e-mail a status - student).
Je nutné psát \B{bez diakritiky}!

\fxnote[author=PŠ]{\textcolor{mygreen}{Sem přijde screenshot formulářes}}

V sekci \B{Product information} pod textem \B{„I already have a Serial Number that starts with 9020“} zaškrtneme možnost \B{No} a do kolonky níže napíšeme \B{9SDK2019}.
Na pravé straně poté zaškrtneme nejnovější verzi, tedy \B{2020-2021}.
Vyplněný formulář odešleme kliknutím na tlačítko \B{Request download}.
Na další stránce potvrdíme licenční podmínky tlačítkem \B{Accept and Continue}.

Nyní jsme se již dostaly na stránku, odkud můžeme SDK stáhnout.
Klikneme tedy na tlačítko \B{Download}, čímž si stáhneme instalátor.
Okno ještě \B{nezavíráme} - budeme z něj potřebovat zkopírovat licenční čísla. 

\subsection{Instalace}
Stažený instalátor otevřeme. 
Objeví se nám okno, ve kterém můžeme nastavit, kam chceme vyextrahovat soubory instalace.
Jakmile máme umístění zvolené, klikneme na tlačítko \B{Unzip}.
Chvíli počkáme a otevře se nám \It{Manažer instalací SOLIDWORKS 2020}.
Pokud se nám objeví okno informující, že po předchozí instalaci nebyl dokončen restart systému, stačí jej odklepnout tlačítkem \B{OK}.
Na obrazovce, kde můžeme zvolit typ instalace ponecháme zaškrtnuté \It{Instalovat na tento počítač} a klikneme na \B{Další}.

Nyní po nás bude instalátor chtít zadat sériová čísla. 
Otevřeme si tedy webový prohlížeč se stránkou, kde byla tato čísla napsaná.


\newpage

% Zaver prace
\input{CHAPTERS/ZAVER.tex}
\newpage

\appendix
\addcontentsline{toc}{chapter}{Přílohy}

% Prilohy
\chapter{Obrazové přílohy}

%\begin{figure}[h]
%    \centering
%    \includegraphics[width=0.85\textwidth]{img/ToBeRemoved/PPSB-T_BOTH.png}
%    \caption{Vizualizace PPSB-T (horní strana vpravo, dolní vlevo).}
%    \label{fig:PPSB-T_VISUAL}
%\end{figure}

\printbibliography[title=Literatura]
\addcontentsline{toc}{chapter}{Literatura}

\listoffigures
\addcontentsline{toc}{section}{Seznam obrázků}

\listoftables
\addcontentsline{toc}{section}{Seznam tabulek}

\end{document}
